\documentclass{article}
\usepackage{graphicx} % Required for inserting images

\title{FDS Exercise 21-oct}
\author{Roberto Magno Mazzotta}
\date{October 2024}

\begin{document}

\maketitle 

\section{exercise on linear regression }


\documentclass{article}
\usepackage{amsmath}

\begin{document}

\title{Normal Equation with Detailed Steps}
\author{}
\date{}
\maketitle

\section*{Dati forniti}

\[
X = [1, 2, 3], \quad y = [2, 3, 5]
\]

L'obiettivo è trovare i parametri \( \theta_0 \) (intercetta) e \( \theta_1 \) (coefficiente angolare) usando l'equazione normale.

\section*{Passo 1: Aggiungere una colonna di 1 a \( X \) per il termine di bias}

Dobbiamo aggiungere una colonna di 1 a \( X \) per includere il termine di bias. La nuova matrice \( X_{\text{bias}} \) sarà:

\[
X_{\text{bias}} = \begin{bmatrix} 1 & 1 \\ 1 & 2 \\ 1 & 3 \end{bmatrix}
\]

\section*{Passo 2: Calcolo di \( X^T X \)}

Ora, calcoliamo il prodotto tra la matrice trasposta di \( X_{\text{bias}} \) e \( X_{\text{bias}} \):

\subsection*{1. Trasposta di \( X_{\text{bias}} \):}
\[
X_{\text{bias}}^T = \begin{bmatrix} 1 & 1 & 1 \\ 1 & 2 & 3 \end{bmatrix}
\]

\subsection*{2. Prodotto tra \( X_{\text{bias}}^T \) e \( X_{\text{bias}} \):}
\[
X^T X = \begin{bmatrix} 1 & 1 & 1 \\ 1 & 2 & 3 \end{bmatrix} \times \begin{bmatrix} 1 & 1 \\ 1 & 2 \\ 1 & 3 \end{bmatrix}
\]
Facciamo i calcoli per ciascun elemento della matrice risultante:

\[
X^T X = \begin{bmatrix} 
1 \times 1 + 1 \times 1 + 1 \times 1 & 1 \times 1 + 1 \times 2 + 1 \times 3 \\
1 \times 1 + 2 \times 1 + 3 \times 1 & 1 \times 1 + 2 \times 2 + 3 \times 3
\end{bmatrix}
\]

\[
X^T X = \begin{bmatrix} 
3 & 6 \\
6 & 14
\end{bmatrix}
\]

\section*{Passo 3: Calcolo dell'inversa di \( X^T X \)}

Ora dobbiamo calcolare l'inversa della matrice \( X^T X \). Per una matrice \( 2 \times 2 \), l'inversa si calcola con la seguente formula:

\[
A^{-1} = \frac{1}{ad - bc} \begin{bmatrix} d & -b \\ -c & a \end{bmatrix}
\]

Applicando questa formula a \( X^T X \):

\[
X^T X = \begin{bmatrix} 3 & 6 \\ 6 & 14 \end{bmatrix}
\]

\subsection*{1. Calcoliamo il determinante:}
\[
\text{det}(X^T X) = (3 \times 14) - (6 \times 6) = 42 - 36 = 6
\]

\subsection*{2. Applichiamo la formula per l'inversa:}
\[
(X^T X)^{-1} = \frac{1}{6} \begin{bmatrix} 14 & -6 \\ -6 & 3 \end{bmatrix}
\]

\subsection*{3. Moltiplichiamo ogni elemento per \( \frac{1}{6} \):}
\[
(X^T X)^{-1} = \begin{bmatrix} \frac{14}{6} & \frac{-6}{6} \\ \frac{-6}{6} & \frac{3}{6} \end{bmatrix} = \begin{bmatrix} 2.33 & -1.00 \\ -1.00 & 0.50 \end{bmatrix}
\]

\section*{Passo 4: Calcolo di \( X^T y \)}

Ora calcoliamo il prodotto tra la matrice trasposta di \( X_{\text{bias}} \) e \( y \):

\[
X^T y = \begin{bmatrix} 1 & 1 & 1 \\ 1 & 2 & 3 \end{bmatrix} \times \begin{bmatrix} 2 \\ 3 \\ 5 \end{bmatrix}
\]

Facciamo i calcoli per ciascun elemento:

\[
X^T y = \begin{bmatrix} 
1 \times 2 + 1 \times 3 + 1 \times 5 \\
1 \times 2 + 2 \times 3 + 3 \times 5
\end{bmatrix}
\]

\[
X^T y = \begin{bmatrix} 10 \\ 23 \end{bmatrix}
\]

\section*{Passo 5: Calcolo di \( \theta \)}

Ora possiamo calcolare i parametri \( \theta \) usando l'equazione normale:

\[
\theta = (X^T X)^{-1} X^T y
\]

Sostituendo i valori calcolati:

\[
\theta = \begin{bmatrix} 2.33 & -1.00 \\ -1.00 & 0.50 \end{bmatrix} \times \begin{bmatrix} 10 \\ 23 \end{bmatrix}
\]

Facciamo i calcoli:

\[
\theta = \begin{bmatrix} 2.33 \times 10 + -1.00 \times 23 \\ -1.00 \times 10 + 0.50 \times 23 \end{bmatrix}
\]

\[
\theta = \begin{bmatrix} 23.3 - 23 \\ -10 + 11.5 \end{bmatrix}
\]

\[
\theta = \begin{bmatrix} 0.33 \\ 1.50 \end{bmatrix}
\]

\section*{Risultato finale}

\[
\theta_0 = 0.33 \quad \text{(intercetta)}
\]
\[
\theta_1 = 1.50 \quad \text{(coefficiente angolare)}
\]

L'equazione della retta che meglio approssima i dati è:

\[
y = 0.33 + 1.50x
\]

\end{document}

\end{document}
